\documentclass{article}
\usepackage[utf8]{inputenc}

\title{NLP Quiz 1}
\author{Alvaro Faundez}
\date{October 2021}

\usepackage{amsmath}
\usepackage{pgfplots}
\pgfplotsset{width=10cm,compat=1.9}
\usepackage{float}
\usepackage{graphicx}
\usetikzlibrary{calc,angles,quotes}
\usepackage{makecell}

\newcommand\given[1][]{\:#1\vert\:}

\begin{document}

\maketitle

\section*{Question 1}

\paragraph{Write out the joint probability for the following sentence using the chain rule:}

\begin{equation*}
    p(\text{There}, \text{is}, \text{only}, \text{one}, \text{person}, \text{who}, \text{is}, \text{not}, \text{ordinary})
\end{equation*}

\paragraph{Write out the probability above using the second-order Markov assumption.}

\subsection*{Answer}

\subsubsection*{Using the chain rule:}

\begin{equation*}
    \begin{split}
    P(\text{There}, \text{is}, \text{only}, \text{one}, \text{person}, \text{who}, \text{is}, \text{not}, \text{ordinary}) =\ & P(\text{There}) \times \\
    & P(\text{is} \given \text{There}) \times \\
    & P(\text{only} \given \text{There}, \text{is} ) \times \\
    & P(\text{one} \given \text{There}, \text{is}, \text{only}) \times \\
    & P(\text{person} \given \text{There}, \text{is}, \text{only}, \text{one}) \times \\
    & P(\text{who} \given \text{There}, \text{is}, \text{only}, \text{one}, \text{person}) \times \\
    & P(\text{is} \given \text{There}, \text{is}, \text{only}, \text{one}, \text{person}, \text{who} \times \\
    & P(\text{not}, \given \text{There}, \text{is}, \text{only}, \text{one}, \text{person}, \text{who}, \text{is}) \times \\
    & P(\text{ordinary} \given \text{There}, \text{is}, \text{only}, \text{one}, \text{person}, \text{who}, \text{is}, \text{not})
    \end{split}
\end{equation*}

\subsubsection*{Using the second-order Markov assumption:}

\begin{equation*}
    \begin{split}
    P(\text{There}, \text{is}, \text{only}, \text{one}, \text{person}, \text{who}, \text{is}, \text{not}, \text{ordinary})) =\ & P(\text{There}) \times \\
    & P(\text{is} \given \text{There}) \times \\
    & P(\text{only} \given \text{There}, \text{is} ) \times \\
    & P(\text{one} \given \text{is}, \text{only}) \times \\
    & P(\text{person} \given  \text{only}, \text{one}) \times \\
    & P(\text{who} \given , \text{one}, \text{person}) \times \\
    & P(\text{is} \given \text{person}, \text{who} \times \\
    & P(\text{not}, \given \text{who}, \text{is}) \times \\
    & P(\text{ordinary}) \given \text{is}, \text{not})
    \end{split}
\end{equation*}

\clearpage

\section*{Question 2}

\paragraph{Consider the following training corpus T of sentences:}

\begin{itemize}
    \item START Karlsson is round STOP
    \item START He lives on the roof STOP
    \item START He is happy STOP
    \item START On the roof STOP
    \item START Karlsson lives happily STOP
\end{itemize}

\paragraph{(b) Compute the following maximum likelihood parameters:}

\begin{equation*}
    \begin{split}
    p(\text{Karlsson} \given \text{START})=\\
    p(\text{Karlsson} \given \text{lives, happily})=\\
    p(\text{STOP}|\text{happy})=
    \end{split}
\end{equation*}

\paragraph{(c) Compute the probability of the following sentences under the trigram model trained on T:}

\begin{equation*}
    \begin{split}
    &\text{START Karlsson is happy STOP}\\
    &\text{START Karlsson lives on the roof STOP}
    \end{split}
\end{equation*}

\subsection*{Answer}

\subsubsection*{(b)}

\begin{equation*}
    \begin{split}
    p(\text{Karlsson} \given \text{START}) = \frac{c(START, Karlsson)}{c(START)} = \frac{2}{5}
    \end{split}
\end{equation*}

\begin{equation*}
    \begin{split}
    p(\text{Karlsson} \given \text{lives, happily}) = \frac{c(lives, happily, Karlsson)}{c(lives, happily)} = \frac{0}{1}
    \end{split}
\end{equation*}


\begin{equation*}
    \begin{split}
    p(\text{STOP} \given \text{happy}) = \frac{c(happy, STOP)}{c(happy)} = \frac{1}{1}
    \end{split}
\end{equation*}

\subsubsection*{(c)}

\paragraph{Sentence "START START Karlsson is happy STOP"}

\begin{equation*}
    \begin{split}
    p(\text{START Karlsson is happy STOP}) =&\ P(Karlsson \given START, START) \times \\
    & P(is \given START, Karlsson) \times \\
    & P(happy \given Karlson, is) \times \\
    & P(STOP \given is, happy)
    \end{split}
\end{equation*}

The probabilities needed are (adding an extra start to each sentence):

\begin{equation*}
    \begin{split}
    P(Karlsson \given START, START) = \frac{c(START, START, Karlsson)}{ c(START, START)} = \frac{2}{5}
    \end{split}
\end{equation*}

\begin{equation*}
    \begin{split}
    P(is \given START, Karlsson) = \frac{c(START, Karlsson, is)}{ c(START, Karlson)} = \frac{1}{2}
    \end{split}
\end{equation*}

\begin{equation*}
    \begin{split}
    P(happy \given Karlson, is) = \frac{c(Karlsson, is, happy)}{ c(Karlson, is)} = \frac{0}{1}
    \end{split}
\end{equation*}

\begin{equation*}
    \begin{split}
    P(STOP \given is, happy) = \frac{c(is, happy, STOP)}{c(is, happy)} = \frac{1}{1}
    \end{split}
\end{equation*}

Then, 

\begin{equation*}
    \begin{split}
    p(\text{START Karlsson is happy STOP}) =\frac{2}{5} \times \frac{1}{2} \times \frac{0}{1} \times \frac{1}{1} = 0
    \end{split}
\end{equation*}

\paragraph{Sentence "START START Karlsson lives on the roof STOP"}

\begin{equation*}
    \begin{split}
    p(\text{START START Karlsson lives on the roof STOP}) =&\ P(Karlsson \given START, START) \times \\
    & P(lives \given START, Karlsson) \times \\
    & P(on \given Karlsson, lives) \times \\
    & P(the \given lives, on) \times \\
    & P(roof \given on, the) \times \\
    & P(STOP \given the, roof)
    \end{split}
\end{equation*}

The probabilities needed are (adding an extra start to each sentence):

\begin{equation*}
    \begin{split}
    P(Karlsson \given START, START) = \frac{c(START, START, Karlsson)}{ c(START, START)} = \frac{2}{5}
    \end{split}
\end{equation*}

\begin{equation*}
    \begin{split}
    P(lives \given START, Karlsson) = \frac{c(START, Karlsson, lives)}{ c(START, Karlsson)} = \frac{1}{2}
    \end{split}
\end{equation*}

\begin{equation*}
    \begin{split}
    P(on \given Karlsson, lives) = \frac{c(Karlsson, lives, on)}{ c( Karlsson, lives)} = \frac{0}{1}
    \end{split}
\end{equation*}

\begin{equation*}
    \begin{split}
    P(the \given lives, on) = \frac{c(lives, on, the)}{ c(lives, on)} = \frac{1}{1}
    \end{split}
\end{equation*}

\begin{equation*}
    \begin{split}
    P(roof \given on, the) = \frac{c(on, the, roof)}{ c(on, the)} = \frac{1}{2}
    \end{split}
\end{equation*}

\begin{equation*}
    \begin{split}
    P(STOP \given the, roof) = \frac{c(the, roof, STOP)}{ c(the, roof)} = \frac{2}{2}
    \end{split}
\end{equation*}


\begin{equation*}
    \begin{split}
    p(\text{START START Karlsson lives on the roof STOP}) = \frac{1}{2} \times \frac{0}{1} \times \frac{1}{1} \times \frac{1}{2} \times \frac{2}{2} = 0
    \end{split}
\end{equation*}






\clearpage

\section*{Question 3}
\paragraph{We have the following training corpus:}

\begin{equation*}
    \begin{split}
    &\text{the green book STOP}\\
    &\text{my blue book STOP}\\
    &\text{his green house STOP}\\
    &\text{book STOP}
    \end{split}
\end{equation*}


\paragraph{Assume we have a language model based on this corpus using linear interpolation with $\lambda_i = 1 / 3$ for all $i$. Compute the value of the parameter $p(book--the\ green)$ under this model. Assume STOP as part of your unigram model.}

\begin{equation*}
p(book\given the\ green)
\end{equation*}

\subsection*{Answer}


\begin{equation*}
p(book \given the, green) = \lambda_1 \times p_{ML}(book \given the, green ) + \lambda_2 \times p_{ML}(book \given green) + \lambda_3 \times p_{ML}(book)
\end{equation*}

\begin{equation*}
 p_{ML}(book \given the, green ) = \frac{c(the, green, book)}{c(the, green)} = \frac{1}{1}
\end{equation*}

\begin{equation*}
 p_{ML}(book \given green ) = \frac{c(green, book)}{c(green)} = \frac{1}{2}
\end{equation*}

\begin{equation*}
 p_{ML}(book ) = \frac{c(book)}{c()} = \frac{3}{14}
\end{equation*}

Then,

\begin{equation*}
p(book \given the green) = (\frac{1}{3} \times 1) + (\frac{1}{3} \times \frac{1}{2}) + (\frac{1}{3} \times \frac{3}{14}) = \frac{1}{3} + \frac{1}{6} + \frac{1}{14} = 0.57142857142
\end{equation*}

\end{document}
